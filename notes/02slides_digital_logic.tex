\pdfminorversion=6
\documentclass[xcolor=table]{beamer}
%\newcommand{\pagelogo}{ru/logo.png}
%\newcommand{\bannerlogo}{ru/logo.png}

\usepackage{svg}
\usepackage{lmodern,amsmath,amssymb}
\usepackage{adjustbox}
\usepackage{mathtools}
\usepackage{algpseudocode}
\usepackage[customcolors]{hf-tikz}
\usetheme{Warsaw}
\setbeamertemplate{enumerate items}[default]
\setbeamerfont{enumerate item}{series=\bfseries\itshape}
\usecolortheme{beaver}
\usefonttheme[onlymath]{serif}
\usepackage[T1]{fontenc} %  fontenc is used to fix the bug for greek letter \Delta
\usepackage{arydshln}
\usepackage{cancel}
\usepackage{blkarray, bigstrut}
\usepackage{accents}
\usepackage{listings}
\usepackage{multicol}
\usepackage{hyperref}
\usepackage{tikz}
\usepackage{graphicx}
\setlength\columnsep{15pt}
\usepackage{makecell}
\renewcommand{\arraystretch}{1.2}


%% customization of beamer environments
%% copyright notice
\setbeamertemplate{footline}{%
	\leavevmode%
	\hbox{\begin{beamercolorbox}[wd=.5\paperwidth,ht=2.0ex,dp=2.125ex,leftskip=.3cm plus1fill,rightskip=.3cm]{author in head/foot}%
			\usebeamerfont{author in head/foot}\copyright Dov Kruger 2024\vspace{-1ex}
		\end{beamercolorbox}%
		\begin{beamercolorbox}[wd=.5\paperwidth,ht=2.0ex,dp=2.125ex,leftskip=.3cm,rightskip=.3cm plus1fil]{title in head/foot}%
			\usebeamerfont{title in head/foot}{\insertshorttitle}\vspace{-1ex}
	\end{beamercolorbox}}%
	\vskip0pt%
}
%% logo banner
\setbeamertemplate{background canvas}{%
	\raisebox{-\paperheight+10pt}[0pt][0pt]{%
		\makebox[\paperwidth][l]{%
			\hspace{1cm}\includegraphics[width=1.5cm]{ru/logo.png}%
		}%
	}%
}
%% create an environment called "withoutheadline" to save space on content slides
\makeatletter
\newenvironment{withoutheadline}{
	\setbeamertemplate{headline}[default]
	\def\beamer@entrycode{\vspace*{-\headheight}}
	\setbeamertemplate{background canvas}{%
		\raisebox{-\paperheight+\headheight+10pt}[0pt][0pt]{%
			\makebox[\paperwidth][l]{%
				\hspace{.5cm}\includegraphics[width=1cm]{ru/logo.png}%
			}%
		}%
	}
}{}
\makeatother
%% create an environment called "withoutheadlinelogoright" in which the logo is located on the right
\makeatletter
\newenvironment{withoutheadlinelogoright}{
	\setbeamertemplate{headline}[default]
	\def\beamer@entrycode{\vspace*{-\headheight}}
	\setbeamertemplate{background canvas}{%
		\raisebox{-\paperheight+\headheight+17pt}[0pt][0pt]{%
			\makebox[\paperwidth][r]{%
				\includegraphics[width=1cm]{ru/logo.png}\hspace{.5cm}%
			}%
		}%
	}
}{}
\makeatother
%% set up beginning pages for each section and subsection
\newif\ifSectionTitlePage
\newcommand*\SectionTitlePagedefault{\SectionTitlePagefalse}

\newcommand\AllSectionsWithTitlePage{%
  \SectionTitlePagetrue
  \renewcommand*\SectionTitlePagedefault{\SectionTitlePagetrue}%
}
\newcommand\AllSectionsWithoutTitlePage{%
  \SectionTitlePagefalse
  \renewcommand*\SectionTitlePagedefault{\SectionTitlePagefalse}%
}
\newcommand\NextSectionWithTitlePage{\SectionTitlePagetrue}
\newcommand\NextSectionWithoutTitlePage{\SectionTitlePagefalse}

\AtBeginSection[]{%
  \ifSectionTitlePage
    \begin{frame}
    \vfill
    \centering
    \begin{beamercolorbox}[sep=8pt,center,shadow=true,rounded=true]{title}
      \usebeamerfont{title}\insertsection\par%
    \end{beamercolorbox}
    \vfill
    \end{frame}
  \fi
  \SectionTitlePagedefault
}
% subsection page
\AtBeginSubsection[]{
	\begin{frame}
		\vfill
		\centering
		\begin{beamercolorbox}[sep=8pt,center,shadow=true,rounded=true]{title}
			\usebeamerfont{subtitle}\insertsubsection\par%
		\end{beamercolorbox}
		\vfill
	\end{frame}
}
\AllSectionsWithTitlePage
% pagenumber
\addtobeamertemplate{navigation symbols}{}{%
	\usebeamerfont{footline}%
	\usebeamercolor[fg]{footline}%
	\hspace{1em}%
	\insertframenumber/\inserttotalframenumber
}
%% enable options for switching theme colors
\definecolor{CraneYellow}{RGB}{252,187,6}
\definecolor{CraneBlack}{RGB}{4,6,76}
\definecolor{CustomBlue}{RGB}{51,51,178}
\definecolor{CustomGreen}{RGB}{50, 205, 50}

\newcommand{\setframecolorExtra}{	
	\setbeamercolor{frametitle}{fg=CraneBlack,bg=CraneYellow}}
\newcommand{\setframecolorDeep}{	
	\setbeamercolor{frametitle}{fg=CraneBlack,bg=CustomGreen!50}}
\newcommand{\setframecolorAct}{
	\setbeamercolor{frametitle}{fg=CraneBlack,bg=CustomBlue!50}}



\usepackage{soul} %for \ul underline command which wraps text while underlining 
\newcommand{\mystep}[1]{{\vspace{2mm}\noindent\textbf{#1}}}
\newcommand{\myhigh}[1]{\textit{\ul{#1}}}
% Robotics Math
\newcommand{\realfield}{\hbox{I \kern -.4em R}}
\newcommand {\mb}[1]{\mathbf{#1}} % all replaced
\newcommand {\bs}[1]{\boldsymbol{#1}}
\newcommand{\uvec}[1]{\hat{\mathbf{#1}}}
\newcommand{\uvecf}[3]{\,^{#1}\hat{\mathbf{#2}}_{#3}}
\newcommand{\T}{^{\top}}  %shortcut for transpose
\newcommand*{\diameter}{\bigcirc\kern-0.95em\diagup}
\newcommand{\rmd}{\textrm{d}}  %shortcut for derivative
% Control Math
\newcommand{\ddtn}[2]{\dfrac{\rmd^{#2} #1}{\rmd t^{#2}}}
\newcommand{\ddt}[1]{\dfrac{\rmd #1}{\rmd t}}
\newcommand{\lap}[1]{\mathcal{L}\left[#1\right]}
\newcommand{\lapinv}[1]{\mathcal{L}^{-1}\left[#1\right]}
%% Remarks and Conclusions index setup
\newcounter{remark}[section]
\newcounter{remarkSavedIndex}[section]
\newcommand{\remarkIndex}{\refstepcounter{remark}\textit{Remark \theremark}}
\newcommand{\remarkSaveIndex}{\setcounter{remarkSavedIndex}{\value{remark}}}
\newcommand{\remarkLoadIndex}{\setcounter{remark}{\value{remarkSavedIndex}}}
\newcounter{conclusion}[section]
\newcounter{conclusionSavedIndex}[section]
\renewcommand{\theconclusion}{\Roman{conclusion}}
\newcommand{\conclusionIndex}{\refstepcounter{conclusion}\textbf{Conclusion (\theconclusion)}}
\newcommand{\conclusionSaveIndex}{\setcounter{conclusionSavedIndex}{\value{conclusion}}}
\newcommand{\conclusionLoadIndex}{\setcounter{conclusion}{\value{conclusionSavedIndex}}}

%% Custom slide options
%% This code contains different options:
% 1. Extended set of slides (including deep dives and extra challenges), named Lect_X_extended
% 2. Lecture presenting slides (trimming off extra challenges), named Lect_X
% 3. Lecture presenting slides with bullet advancing mode, named Lect_X_pres

\ifdefined\slidePresentingMode
\else
	\newcommand\slidePresentingMode{1}
\fi

\newif\ifSlideExtra
\newif\ifSlideDeepDive
\newif\ifSlideBulletAdvance

\ifnum \slidePresentingMode=1
	\SlideExtratrue
	\SlideDeepDivetrue
	\SlideBulletAdvancefalse	
\fi

\ifnum \slidePresentingMode=2
	\SlideExtrafalse
	\SlideDeepDivefalse
	\SlideBulletAdvancefalse
\fi

\ifnum \slidePresentingMode=3
	\SlideExtrafalse
	\SlideDeepDivefalse
	\SlideBulletAdvancetrue
\fi

%% activate the following line to generate presenting slides as oppose to notes slide
\ifSlideBulletAdvance
	\beamerdefaultoverlayspecification{<+->}
\fi



%% custom slide indices
% define an index for active learning act
\newcounter{activeLearn}
\newcommand{\activeLearnIndex}{\refstepcounter{activeLearn}In-class Exercise \#\theactiveLearn }
% define an index for deep dives
\newcounter{deepDive}
\newcommand{\deepDiveIndex}{\refstepcounter{deepDive}Deep Dive \#\thedeepDive }
% define an index for extra challenges
\newcounter{extraChallenge}
\newcommand{\extraChIndex}{\refstepcounter{extraChallenge}Extra Challenge \#\theextraChallenge }

%% table preamble
%\newcolumntype{C}[1]{>{\centering\arraybackslash}m{#1}}

%% Dov's requests for common spacing commands on slides
\newcommand{\smallspace}{\vspace{2mm}}
\newcommand{\bigspace}{\vspace{5mm}}

\institute{Department of Electrical and Computer Engineering\\Rutgers University}
% logo of my university
\titlegraphic{\centering\includegraphics[width=1cm]{ru/logo.png}
}
\author{Dov Kruger}
\date{\today}


\title[]{Introduction to Digital Logic}
\begin{document}
\begin{frame}
\titlepage
\end{frame}

\begin{withoutheadline}

\begin{frame}{Why Digital Logic in a Cryptography Lesson?}
\begin{itemize}
    \item All modern cryptography is implemented on computer
    \item We will look at cryptographic building blocks and how they are implemented
    \item Vital to understand how data is represented on a computer
\end{itemize}
\end{frame}
    
\begin{frame}{Bits and Bytes}
\begin{itemize}
    \item Bit: Smallest unit of data (0,1)
    \item Byte: Group of 8 bits
    \item Type of data completely depends on context
\end{itemize}
\end{frame}

\begin{frame}{Data Representations?}
\begin{itemize}
    \item 2 bits: 00, 01, 10, 11
    \item 3 bits: 000, 001, 010, 011, 100, 101, 110, 111
    \item n bits = $2^n$ combinations
    \item Data types
   \begin{itemize}
       \item integers
       \item floating point
       \item ASCII characters
       \item unicode
   \end{itemize}
\end{itemize}
\end{frame}

% \begin{frame}{3 bit}
% \begin{itemize}
%     \item 2 bits: 00, 01, 10, 11
%     \item 3 bits: 000, 001, 010, 011, 100, 101, 110, 111
%     \item n bits = $2^n$ combinations
%     \item Data types
%     \begin{itemize}
%         \item integers
%         \item floating point
%         \item ASCII characters
%         \item unicode
%     \end{itemize}
% \end{itemize}
% \end{frame}

\begin{frame}[fragile]{Representing 3 bit Integers}
% \begin{tabular}{p{2.5cm}|p{2.5cm}|p{2.5cm}}    \hline
%     bits & unsigned & signed \\
%     000  &        0 &      0 \\
%     001  &        1 &      1 \\
%     010  &        2 &      2 \\
%     011  &        3 &      3 \\
%     100  &        4 &     -4 \\
%     101  &        5 &     -3 \\
%     110  &        6 &     -2 \\
%     111  &        7 &     -1 \\ \hline
% \end{tabular}
\end{frame}

\begin{frame}[fragile]{Base 16: Hexadecimal}
\begin{tabular}{c|c|c|c|c}
    \hline
    Hex & bits & --- & Hex & bits  \\ \hline
    0 & 0000 & & 8 & 1000 \\
    1 & 0001 & & 9 & 1001 \\
    2 & 0010 & & A & 1010 \\
    3 & 0011 & & B & 1011 \\
    4 & 0100 & & C & 1100 \\
    5 & 0101 & & D & 1101 \\
    6 & 0110 & & E & 1110 \\
    7 & 0111 & & F & 1111 \\    \hline
\end{tabular}
\end{frame}

\begin{frame}{Hexadecimal System}
\begin{itemize}
    \item Hex: Base-16 system
    \item Compare with binary and decimal
    \item Easier readability of binary data
    \item Converting between hex, binary, and decimal
\end{itemize}
\end{frame}
    
\begin{frame}{Encoding a byte in Hexadecimal}
\begin{itemize}
    \item D9 = 10011001
    \item AF = 10101111
    \item 8C = 10001100
\end{itemize}
\end{frame}

\begin{frame}{ASCII Encoding}
\begin{itemize}
    \item ASCII: American Standard Code for Information Interchange
    \item Maps characters to binary values
    \item ASCII table overview \url{https://www.ascii-code.com/}
    \item Example: 'A' = 01000001 = 65
    \item Example: 'B' = 01000010 = 66
    \item Example: 'a' = 01100001 = 97
\end{itemize}
\end{frame}
    

\begin{frame}[fragile]{Bitwise Operations}
\begin{itemize}
    \item Logic gates on computers are built out of transistors
    \item Circuits are built to yield desired operations (add, subtract, multiply, divide)
\end{itemize}
\begin{tabular}{c|c|c|c|c|c}    %\hline
    A & B & NOT A & A AND B & A OR B & A XOR B \\ %\hline
    0 & 0 &     1 &       0 &      0 &       0 \\ %\hline
    0 & 1 &     1 &       0 &      1 &       1 \\ %\hline
    1 & 0 &     0 &       0 &      1 &       1 \\ %\hline
    1 & 1 &     0 &       1 &      1 &       0 \\ \hline
\end{tabular}
\end{frame}


\begin{frame}{Computer Bit Operations}
\begin{itemize}
    \item Bitwise operations
\end{itemize}
\end{frame}

\end{withoutheadline}
\end{document} 
