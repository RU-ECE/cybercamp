\pdfminorversion=6
\documentclass[xcolor=table]{beamer}
%\newcommand{\pagelogo}{ru/logo.png}
%\newcommand{\bannerlogo}{ru/logo.png}

\usepackage{svg}
\usepackage{lmodern,amsmath,amssymb}
\usepackage{adjustbox}
\usepackage{mathtools}
\usepackage{algpseudocode}
\usepackage[customcolors]{hf-tikz}
\usetheme{Warsaw}
\setbeamertemplate{enumerate items}[default]
\setbeamerfont{enumerate item}{series=\bfseries\itshape}
\usecolortheme{beaver}
\usefonttheme[onlymath]{serif}
\usepackage[T1]{fontenc} %  fontenc is used to fix the bug for greek letter \Delta
\usepackage{arydshln}
\usepackage{cancel}
\usepackage{blkarray, bigstrut}
\usepackage{accents}
\usepackage{listings}
\usepackage{multicol}
\usepackage{hyperref}
\usepackage{tikz}
\usepackage{graphicx}
\setlength\columnsep{15pt}
\usepackage{makecell}
\renewcommand{\arraystretch}{1.2}


%% customization of beamer environments
%% copyright notice
\setbeamertemplate{footline}{%
	\leavevmode%
	\hbox{\begin{beamercolorbox}[wd=.5\paperwidth,ht=2.0ex,dp=2.125ex,leftskip=.3cm plus1fill,rightskip=.3cm]{author in head/foot}%
			\usebeamerfont{author in head/foot}\copyright Dov Kruger 2024\vspace{-1ex}
		\end{beamercolorbox}%
		\begin{beamercolorbox}[wd=.5\paperwidth,ht=2.0ex,dp=2.125ex,leftskip=.3cm,rightskip=.3cm plus1fil]{title in head/foot}%
			\usebeamerfont{title in head/foot}{\insertshorttitle}\vspace{-1ex}
	\end{beamercolorbox}}%
	\vskip0pt%
}
%% logo banner
\setbeamertemplate{background canvas}{%
	\raisebox{-\paperheight+10pt}[0pt][0pt]{%
		\makebox[\paperwidth][l]{%
			\hspace{1cm}\includegraphics[width=1.5cm]{ru/logo.png}%
		}%
	}%
}
%% create an environment called "withoutheadline" to save space on content slides
\makeatletter
\newenvironment{withoutheadline}{
	\setbeamertemplate{headline}[default]
	\def\beamer@entrycode{\vspace*{-\headheight}}
	\setbeamertemplate{background canvas}{%
		\raisebox{-\paperheight+\headheight+10pt}[0pt][0pt]{%
			\makebox[\paperwidth][l]{%
				\hspace{.5cm}\includegraphics[width=1cm]{ru/logo.png}%
			}%
		}%
	}
}{}
\makeatother
%% create an environment called "withoutheadlinelogoright" in which the logo is located on the right
\makeatletter
\newenvironment{withoutheadlinelogoright}{
	\setbeamertemplate{headline}[default]
	\def\beamer@entrycode{\vspace*{-\headheight}}
	\setbeamertemplate{background canvas}{%
		\raisebox{-\paperheight+\headheight+17pt}[0pt][0pt]{%
			\makebox[\paperwidth][r]{%
				\includegraphics[width=1cm]{ru/logo.png}\hspace{.5cm}%
			}%
		}%
	}
}{}
\makeatother
%% set up beginning pages for each section and subsection
\newif\ifSectionTitlePage
\newcommand*\SectionTitlePagedefault{\SectionTitlePagefalse}

\newcommand\AllSectionsWithTitlePage{%
  \SectionTitlePagetrue
  \renewcommand*\SectionTitlePagedefault{\SectionTitlePagetrue}%
}
\newcommand\AllSectionsWithoutTitlePage{%
  \SectionTitlePagefalse
  \renewcommand*\SectionTitlePagedefault{\SectionTitlePagefalse}%
}
\newcommand\NextSectionWithTitlePage{\SectionTitlePagetrue}
\newcommand\NextSectionWithoutTitlePage{\SectionTitlePagefalse}

\AtBeginSection[]{%
  \ifSectionTitlePage
    \begin{frame}
    \vfill
    \centering
    \begin{beamercolorbox}[sep=8pt,center,shadow=true,rounded=true]{title}
      \usebeamerfont{title}\insertsection\par%
    \end{beamercolorbox}
    \vfill
    \end{frame}
  \fi
  \SectionTitlePagedefault
}
% subsection page
\AtBeginSubsection[]{
	\begin{frame}
		\vfill
		\centering
		\begin{beamercolorbox}[sep=8pt,center,shadow=true,rounded=true]{title}
			\usebeamerfont{subtitle}\insertsubsection\par%
		\end{beamercolorbox}
		\vfill
	\end{frame}
}
\AllSectionsWithTitlePage
% pagenumber
\addtobeamertemplate{navigation symbols}{}{%
	\usebeamerfont{footline}%
	\usebeamercolor[fg]{footline}%
	\hspace{1em}%
	\insertframenumber/\inserttotalframenumber
}
%% enable options for switching theme colors
\definecolor{CraneYellow}{RGB}{252,187,6}
\definecolor{CraneBlack}{RGB}{4,6,76}
\definecolor{CustomBlue}{RGB}{51,51,178}
\definecolor{CustomGreen}{RGB}{50, 205, 50}

\newcommand{\setframecolorExtra}{	
	\setbeamercolor{frametitle}{fg=CraneBlack,bg=CraneYellow}}
\newcommand{\setframecolorDeep}{	
	\setbeamercolor{frametitle}{fg=CraneBlack,bg=CustomGreen!50}}
\newcommand{\setframecolorAct}{
	\setbeamercolor{frametitle}{fg=CraneBlack,bg=CustomBlue!50}}



\usepackage{soul} %for \ul underline command which wraps text while underlining 
\newcommand{\mystep}[1]{{\vspace{2mm}\noindent\textbf{#1}}}
\newcommand{\myhigh}[1]{\textit{\ul{#1}}}
% Robotics Math
\newcommand{\realfield}{\hbox{I \kern -.4em R}}
\newcommand {\mb}[1]{\mathbf{#1}} % all replaced
\newcommand {\bs}[1]{\boldsymbol{#1}}
\newcommand{\uvec}[1]{\hat{\mathbf{#1}}}
\newcommand{\uvecf}[3]{\,^{#1}\hat{\mathbf{#2}}_{#3}}
\newcommand{\T}{^{\top}}  %shortcut for transpose
\newcommand*{\diameter}{\bigcirc\kern-0.95em\diagup}
\newcommand{\rmd}{\textrm{d}}  %shortcut for derivative
% Control Math
\newcommand{\ddtn}[2]{\dfrac{\rmd^{#2} #1}{\rmd t^{#2}}}
\newcommand{\ddt}[1]{\dfrac{\rmd #1}{\rmd t}}
\newcommand{\lap}[1]{\mathcal{L}\left[#1\right]}
\newcommand{\lapinv}[1]{\mathcal{L}^{-1}\left[#1\right]}
%% Remarks and Conclusions index setup
\newcounter{remark}[section]
\newcounter{remarkSavedIndex}[section]
\newcommand{\remarkIndex}{\refstepcounter{remark}\textit{Remark \theremark}}
\newcommand{\remarkSaveIndex}{\setcounter{remarkSavedIndex}{\value{remark}}}
\newcommand{\remarkLoadIndex}{\setcounter{remark}{\value{remarkSavedIndex}}}
\newcounter{conclusion}[section]
\newcounter{conclusionSavedIndex}[section]
\renewcommand{\theconclusion}{\Roman{conclusion}}
\newcommand{\conclusionIndex}{\refstepcounter{conclusion}\textbf{Conclusion (\theconclusion)}}
\newcommand{\conclusionSaveIndex}{\setcounter{conclusionSavedIndex}{\value{conclusion}}}
\newcommand{\conclusionLoadIndex}{\setcounter{conclusion}{\value{conclusionSavedIndex}}}

%% Custom slide options
%% This code contains different options:
% 1. Extended set of slides (including deep dives and extra challenges), named Lect_X_extended
% 2. Lecture presenting slides (trimming off extra challenges), named Lect_X
% 3. Lecture presenting slides with bullet advancing mode, named Lect_X_pres

\ifdefined\slidePresentingMode
\else
	\newcommand\slidePresentingMode{1}
\fi

\newif\ifSlideExtra
\newif\ifSlideDeepDive
\newif\ifSlideBulletAdvance

\ifnum \slidePresentingMode=1
	\SlideExtratrue
	\SlideDeepDivetrue
	\SlideBulletAdvancefalse	
\fi

\ifnum \slidePresentingMode=2
	\SlideExtrafalse
	\SlideDeepDivefalse
	\SlideBulletAdvancefalse
\fi

\ifnum \slidePresentingMode=3
	\SlideExtrafalse
	\SlideDeepDivefalse
	\SlideBulletAdvancetrue
\fi

%% activate the following line to generate presenting slides as oppose to notes slide
\ifSlideBulletAdvance
	\beamerdefaultoverlayspecification{<+->}
\fi



%% custom slide indices
% define an index for active learning act
\newcounter{activeLearn}
\newcommand{\activeLearnIndex}{\refstepcounter{activeLearn}In-class Exercise \#\theactiveLearn }
% define an index for deep dives
\newcounter{deepDive}
\newcommand{\deepDiveIndex}{\refstepcounter{deepDive}Deep Dive \#\thedeepDive }
% define an index for extra challenges
\newcounter{extraChallenge}
\newcommand{\extraChIndex}{\refstepcounter{extraChallenge}Extra Challenge \#\theextraChallenge }

%% table preamble
%\newcolumntype{C}[1]{>{\centering\arraybackslash}m{#1}}

%% Dov's requests for common spacing commands on slides
\newcommand{\smallspace}{\vspace{2mm}}
\newcommand{\bigspace}{\vspace{5mm}}

\institute{Department of Electrical and Computer Engineering\\Rutgers University}
% logo of my university
\titlegraphic{\centering\includegraphics[width=1cm]{ru/logo.png}
}
\author{Dov Kruger}
\date{\today}


\title[]{Cryptographic Attacks}
\begin{document}
\begin{frame}
\titlepage
\end{frame}

\begin{withoutheadline}

\begin{frame}{Cryptographic Attacks}
\begin{itemize}
    \item Cryptography is different than other areas of computer science
    \item Bugs are not merely passive
    \item People actively try to find bugs and exploit them to achieve results
\end{itemize}
\end{frame}

\begin{frame}{Cryptographic Attacks}
    \item Evil maid
    \item Plaintext
    \item 
    \item
    \item
    \item
    \item
\end{frame}

\begin{frame}{Plaintext attack}
    \item Given knowledge of a message contents
    \item Reconstruct the key and use it to break other messages
    \item Example:
    \begin{itemize}
        \item message: 00 00 00 00 65 66 67 68
        \item XOR encryption, key 4B
        \item Compute the message
        \item Can you determine the key from the encrypted message?
    \end{itemize}
\end{frame}

\begin{frame}{Chosen Plaintext attack}
    \item Given knowledge of a cryptosystem
    \item Get the user to send a special constructed message
    \item Example for XOR: get the user to send a zero in a known location
\end{frame}

\begin{frame}{RSA Plaintext attack}
\begin{itemize}
    \item More complicated than the previous
    \item RSA can be broken by a plaintext attack
    \item RSA is also computationally expensive
    \item Therefore: we only send random numbers through RSA
    \item Use the Random numbers to construct a joint key (Diffie-Hellman)
    \item Use AES-256 to encrypt the actual message
\end{itemize}
\end{frame}

\begin{frame}{DNS attack}
\begin{itemize}
    \item Redirect lookup of domain names
    \item Build a malicious server that will mimick the real one
    \item Direct the computer to a fake certificate Authority
    \item All access to named web servers can then be faked
\end{itemize}
\end{frame}

\begin{frame}{VPN}
\begin{itemize}
    \item A VPN uses encryption (typically AES-256)
    \item Stops local ISP from spying on your activity
    \item even if you use https, they can see which IP addresses you contact
    \item Safer, particularly if you don't trust the network (hotels, foreign countries)
    \item VPNs: NORDVPN, Surfshark, ExpressVPN, ProtonVPN, PIA 
    \item The VPN insiders themselves can spy on you
    \item They can be forced to give the information to a government
\end{itemize}
\end{frame}

\begin{frame}{Tunnelvision Attack}
\begin{itemize}
    \item Recent vulnerability showed it is possible to fool systems (Android)
    \item They bypass the VPN, making it useless
    \item \url{https://www.wired.com/story/tunnelvision-vpn-attack/}
\end{itemize}
\end{frame}

\begin{frame}{Chosen Plaintext attack}
    \item Given knowledge of a cryptosystem
    \item Get the user to send a special constructed message
    \item Example for XOR: get the user to send a zero in a known location
\end{frame}

\begin{frame}{Side-Channel Attacks}
\begin{itemize}
    \item Encryption uses computing resources
    \item Computers leak information in various ways
    \item Side-channel attacks are used to exploit these
    \item Power
    \item Radio-Frequency Radiation
    \item Cache
    \item Timing
    \item Differential-Fault
    \item Data remanence
\end{itemize}
\url{https://en.wikipedia.org/wiki/Side-channel_attack}
\end{frame}

\begin{frame}{Social Engineering Attacks}
\begin{itemize}
    \item Phishing
    \item Ransomeware
    \item Social Engineering attacks rely on tricking users into initiating the attack
    \begin{itemize}
        \item Clicking on a link
        \item Running a program
        \item Taking an action based on a fraudulent email
    \end{itemize}
\end{itemize}
\end{frame}

\begin{frame}{Repudiation}
\begin{itemize}
    \item A user does something (buys merchandise, sends a bitcoin)
    \item Claims it did not happen
    \item A secure cryptosystem should be able to prevent a dishonest user from claiming it never happened
\end{itemize}
\end{frame}

\begin{frame}{Replay Attack}
\begin{itemize}
    \item A conversation is encrypted and the attacker cannot modify it
    \item Still possible to cause trouble
    \item Example: I order a pizza online
    \item Replay attack: eavesdropper records my encrypted message, and sends it 1 million times
    \item 1 million pizzas are billed to me (and appear at my door!)
\end{itemize}
\end{frame}

\begin{frame}{Dictionary Attack}
\begin{itemize}
        \item Password files today are usually hashed for security
    \item This prevents someone with access from stealing all passwords
    \item If an attack can gain access to the password file, they can try to crack by
    \begin{itemize}
        \item Start with a dictionary of all common password choices
        \item words, common stupid passwords
        \item Hash all words in the dictionary
        \item Compare to the passwords in the file
    \end{itemize}
\end{itemize}
\end{frame}

%Breaking RSA using Schorr's algorithm and a quantum computer
%    The NSA has ordered all ISPs to start transitioning to a new cryptosystem.
%    RSA will not be safe when there are quantum computers with 400 qbits capable of staying up long enough.
%    This has not been done yet, but when it is done, suddenly https:// would no longer be safe.




\end{withoutheadline}
\end{document} 
