\pdfminorversion=6
\documentclass[xcolor=table]{beamer}
%\newcommand{\pagelogo}{ru/logo.png}
%\newcommand{\bannerlogo}{ru/logo.png}

\usepackage{svg}
\usepackage{lmodern,amsmath,amssymb}
\usepackage{adjustbox}
\usepackage{mathtools}
\usepackage{algpseudocode}
\usepackage[customcolors]{hf-tikz}
\usetheme{Warsaw}
\setbeamertemplate{enumerate items}[default]
\setbeamerfont{enumerate item}{series=\bfseries\itshape}
\usecolortheme{beaver}
\usefonttheme[onlymath]{serif}
\usepackage[T1]{fontenc} %  fontenc is used to fix the bug for greek letter \Delta
\usepackage{arydshln}
\usepackage{cancel}
\usepackage{blkarray, bigstrut}
\usepackage{accents}
\usepackage{listings}
\usepackage{multicol}
\usepackage{hyperref}
\usepackage{tikz}
\usepackage{graphicx}
\setlength\columnsep{15pt}
\usepackage{makecell}
\renewcommand{\arraystretch}{1.2}


%% customization of beamer environments
%% copyright notice
\setbeamertemplate{footline}{%
	\leavevmode%
	\hbox{\begin{beamercolorbox}[wd=.5\paperwidth,ht=2.0ex,dp=2.125ex,leftskip=.3cm plus1fill,rightskip=.3cm]{author in head/foot}%
			\usebeamerfont{author in head/foot}\copyright Dov Kruger 2024\vspace{-1ex}
		\end{beamercolorbox}%
		\begin{beamercolorbox}[wd=.5\paperwidth,ht=2.0ex,dp=2.125ex,leftskip=.3cm,rightskip=.3cm plus1fil]{title in head/foot}%
			\usebeamerfont{title in head/foot}{\insertshorttitle}\vspace{-1ex}
	\end{beamercolorbox}}%
	\vskip0pt%
}
%% logo banner
\setbeamertemplate{background canvas}{%
	\raisebox{-\paperheight+10pt}[0pt][0pt]{%
		\makebox[\paperwidth][l]{%
			\hspace{1cm}\includegraphics[width=1.5cm]{ru/logo.png}%
		}%
	}%
}
%% create an environment called "withoutheadline" to save space on content slides
\makeatletter
\newenvironment{withoutheadline}{
	\setbeamertemplate{headline}[default]
	\def\beamer@entrycode{\vspace*{-\headheight}}
	\setbeamertemplate{background canvas}{%
		\raisebox{-\paperheight+\headheight+10pt}[0pt][0pt]{%
			\makebox[\paperwidth][l]{%
				\hspace{.5cm}\includegraphics[width=1cm]{ru/logo.png}%
			}%
		}%
	}
}{}
\makeatother
%% create an environment called "withoutheadlinelogoright" in which the logo is located on the right
\makeatletter
\newenvironment{withoutheadlinelogoright}{
	\setbeamertemplate{headline}[default]
	\def\beamer@entrycode{\vspace*{-\headheight}}
	\setbeamertemplate{background canvas}{%
		\raisebox{-\paperheight+\headheight+17pt}[0pt][0pt]{%
			\makebox[\paperwidth][r]{%
				\includegraphics[width=1cm]{ru/logo.png}\hspace{.5cm}%
			}%
		}%
	}
}{}
\makeatother
%% set up beginning pages for each section and subsection
\newif\ifSectionTitlePage
\newcommand*\SectionTitlePagedefault{\SectionTitlePagefalse}

\newcommand\AllSectionsWithTitlePage{%
  \SectionTitlePagetrue
  \renewcommand*\SectionTitlePagedefault{\SectionTitlePagetrue}%
}
\newcommand\AllSectionsWithoutTitlePage{%
  \SectionTitlePagefalse
  \renewcommand*\SectionTitlePagedefault{\SectionTitlePagefalse}%
}
\newcommand\NextSectionWithTitlePage{\SectionTitlePagetrue}
\newcommand\NextSectionWithoutTitlePage{\SectionTitlePagefalse}

\AtBeginSection[]{%
  \ifSectionTitlePage
    \begin{frame}
    \vfill
    \centering
    \begin{beamercolorbox}[sep=8pt,center,shadow=true,rounded=true]{title}
      \usebeamerfont{title}\insertsection\par%
    \end{beamercolorbox}
    \vfill
    \end{frame}
  \fi
  \SectionTitlePagedefault
}
% subsection page
\AtBeginSubsection[]{
	\begin{frame}
		\vfill
		\centering
		\begin{beamercolorbox}[sep=8pt,center,shadow=true,rounded=true]{title}
			\usebeamerfont{subtitle}\insertsubsection\par%
		\end{beamercolorbox}
		\vfill
	\end{frame}
}
\AllSectionsWithTitlePage
% pagenumber
\addtobeamertemplate{navigation symbols}{}{%
	\usebeamerfont{footline}%
	\usebeamercolor[fg]{footline}%
	\hspace{1em}%
	\insertframenumber/\inserttotalframenumber
}
%% enable options for switching theme colors
\definecolor{CraneYellow}{RGB}{252,187,6}
\definecolor{CraneBlack}{RGB}{4,6,76}
\definecolor{CustomBlue}{RGB}{51,51,178}
\definecolor{CustomGreen}{RGB}{50, 205, 50}

\newcommand{\setframecolorExtra}{	
	\setbeamercolor{frametitle}{fg=CraneBlack,bg=CraneYellow}}
\newcommand{\setframecolorDeep}{	
	\setbeamercolor{frametitle}{fg=CraneBlack,bg=CustomGreen!50}}
\newcommand{\setframecolorAct}{
	\setbeamercolor{frametitle}{fg=CraneBlack,bg=CustomBlue!50}}



\usepackage{soul} %for \ul underline command which wraps text while underlining 
\newcommand{\mystep}[1]{{\vspace{2mm}\noindent\textbf{#1}}}
\newcommand{\myhigh}[1]{\textit{\ul{#1}}}
% Robotics Math
\newcommand{\realfield}{\hbox{I \kern -.4em R}}
\newcommand {\mb}[1]{\mathbf{#1}} % all replaced
\newcommand {\bs}[1]{\boldsymbol{#1}}
\newcommand{\uvec}[1]{\hat{\mathbf{#1}}}
\newcommand{\uvecf}[3]{\,^{#1}\hat{\mathbf{#2}}_{#3}}
\newcommand{\T}{^{\top}}  %shortcut for transpose
\newcommand*{\diameter}{\bigcirc\kern-0.95em\diagup}
\newcommand{\rmd}{\textrm{d}}  %shortcut for derivative
% Control Math
\newcommand{\ddtn}[2]{\dfrac{\rmd^{#2} #1}{\rmd t^{#2}}}
\newcommand{\ddt}[1]{\dfrac{\rmd #1}{\rmd t}}
\newcommand{\lap}[1]{\mathcal{L}\left[#1\right]}
\newcommand{\lapinv}[1]{\mathcal{L}^{-1}\left[#1\right]}
%% Remarks and Conclusions index setup
\newcounter{remark}[section]
\newcounter{remarkSavedIndex}[section]
\newcommand{\remarkIndex}{\refstepcounter{remark}\textit{Remark \theremark}}
\newcommand{\remarkSaveIndex}{\setcounter{remarkSavedIndex}{\value{remark}}}
\newcommand{\remarkLoadIndex}{\setcounter{remark}{\value{remarkSavedIndex}}}
\newcounter{conclusion}[section]
\newcounter{conclusionSavedIndex}[section]
\renewcommand{\theconclusion}{\Roman{conclusion}}
\newcommand{\conclusionIndex}{\refstepcounter{conclusion}\textbf{Conclusion (\theconclusion)}}
\newcommand{\conclusionSaveIndex}{\setcounter{conclusionSavedIndex}{\value{conclusion}}}
\newcommand{\conclusionLoadIndex}{\setcounter{conclusion}{\value{conclusionSavedIndex}}}

%% Custom slide options
%% This code contains different options:
% 1. Extended set of slides (including deep dives and extra challenges), named Lect_X_extended
% 2. Lecture presenting slides (trimming off extra challenges), named Lect_X
% 3. Lecture presenting slides with bullet advancing mode, named Lect_X_pres

\ifdefined\slidePresentingMode
\else
	\newcommand\slidePresentingMode{1}
\fi

\newif\ifSlideExtra
\newif\ifSlideDeepDive
\newif\ifSlideBulletAdvance

\ifnum \slidePresentingMode=1
	\SlideExtratrue
	\SlideDeepDivetrue
	\SlideBulletAdvancefalse	
\fi

\ifnum \slidePresentingMode=2
	\SlideExtrafalse
	\SlideDeepDivefalse
	\SlideBulletAdvancefalse
\fi

\ifnum \slidePresentingMode=3
	\SlideExtrafalse
	\SlideDeepDivefalse
	\SlideBulletAdvancetrue
\fi

%% activate the following line to generate presenting slides as oppose to notes slide
\ifSlideBulletAdvance
	\beamerdefaultoverlayspecification{<+->}
\fi



%% custom slide indices
% define an index for active learning act
\newcounter{activeLearn}
\newcommand{\activeLearnIndex}{\refstepcounter{activeLearn}In-class Exercise \#\theactiveLearn }
% define an index for deep dives
\newcounter{deepDive}
\newcommand{\deepDiveIndex}{\refstepcounter{deepDive}Deep Dive \#\thedeepDive }
% define an index for extra challenges
\newcounter{extraChallenge}
\newcommand{\extraChIndex}{\refstepcounter{extraChallenge}Extra Challenge \#\theextraChallenge }

%% table preamble
%\newcolumntype{C}[1]{>{\centering\arraybackslash}m{#1}}

%% Dov's requests for common spacing commands on slides
\newcommand{\smallspace}{\vspace{2mm}}
\newcommand{\bigspace}{\vspace{5mm}}

\institute{Department of Electrical and Computer Engineering\\Rutgers University}
% logo of my university
\titlegraphic{\centering\includegraphics[width=1cm]{ru/logo.png}
}
\author{Dov Kruger}
\date{\today}


\title[]{Introduction to Cryptography}
\begin{document}
\begin{frame}
\titlepage
\end{frame}

\begin{withoutheadline}

\begin{frame}{By the end of today you will be able to...}
\begin{itemize}
    \item Define major cryptographic concepts and terms
    \item Generate problems that students can solve interactively as puzzles
    \item Generate programming exercises to teach how to do real cryptography
    \item Develop materials for K-12 lessons
    \item Support differential lessons (programming and conceptual)
    \item Describe the fundamental computing hardware that modern cryptography uses
    \item Identify weaknesses and dangers in current cryptographic systems
    \item There is presumably a wide range of programming abilities here! You can
    \begin{itemize}
        \item Write lesson plans that avoid programming if that's better for you or your students
        \item Write code using Python or Java, whatever you are most comfortable with
        \item Use ChatGPT or (recommended) Claude to write code that you can use
    \end{itemize}
\end{itemize}
\end{frame}
        
\begin{frame}{Introduction to Cryptography}
\begin{itemize}
    \item What is Cryptography?
    \item Importance of Cryptography
    \item Basic Terminology
    \item Historical Background
    \item Substitution Ciphers
    \item Symmetric Cryptography
    \item Asymmetric Cryptography
    \item Common Cryptographic Algorithms
    \item Cryptographic Applications
    \item Future of Cryptography
\end{itemize}
\end{frame}

\begin{frame}{What is Cryptography?}
\begin{itemize}
    \item Science of securing communication
    \item Encodes messages to protect information
    \item Ensures confidentiality, integrity, authenticity
\end{itemize}
\end{frame}

\begin{frame}{Importance of Cryptography}
\begin{itemize}
    \item Protects sensitive information
    \item Secures online transactions
    \item Prevents unauthorized access
    \item Ensures data integrity
    \item Vital for national security
\end{itemize}
\end{frame}

\begin{frame}{Basic Terminology}
\begin{itemize}
    \item Plaintext: Unencrypted information
    \item Ciphertext: Encrypted information
    \item Encryption: Process of converting plaintext to ciphertext
    \item Decryption: Process of converting ciphertext back to plaintext
    \item Key: Secret value used in encryption/decryption
\end{itemize}
\end{frame}

\begin{frame}{History of Cryptography}
\begin{itemize}
    \item Ancient civilizations
    \item Caesar Cipher
    \item Vigenère Cipher
    \item Playfair Cipher
    \item Enigma Machine
\end{itemize}
\end{frame}

\begin{frame}{Caesar Cipher}
\begin{itemize}
    \item Simple substitution cipher
    \item Shift each letter by fixed number
    \item Example: Shift by 3
    \item Plaintext: HELLO
    \item Ciphertext: KHOOR
\end{itemize}

%\begin{lstlisting}[language=c++,mathescape=true]
%    void f() {}
%\end{lstlisting}
A B C D E F G H I J K L M N O P Q R S T U V W X Y Z \\
D E F G H I J K L M N O P Q R S T U V W X Y Z A B C \\

\end{frame}

\begin{frame}{Caesar Cipher Exercise}
\begin{itemize}
    \item Decode the message
    \item Ciphertext: ZKHQ WKH PRRQ KLWV BRXU HBH
    \item Shift by 3
    \item Plaintext:  \_\_\_\_ \_\_\_ \_\_\_\_ \_\_\_\_ \_\_\_\_ \_\_\_
\end{itemize}
\end{frame}

\begin{frame}{Caesar Cipher Exercise}
\begin{itemize}
    \item Define a secret message
    \item Define the key (shift by an amount from 1 to 26)
    \item Compute the encrypted message
    \item Exchange secret messages with a partner
    \item Decrypt by reversing the process
\end{itemize}
\end{frame}

\begin{frame}{Breaking a Caesar Cipher}
\begin{itemize}
    \item Brute-force attack
    \item Test all possible shifts (25 keys)
    \item Find readable plaintext
    \item Example: Shift by 1 to 25
    \item Q: How can we create a cipher that is more difficult to break?
\end{itemize}
\end{frame}

\begin{frame}{Vigenère Cipher Explained}
\begin{itemize}
    \item Polyalphabetic substitution cipher
    \item Uses a repeating keyword
    \item Each letter in plaintext shifted by keyword letter
    \item Example:
    \begin{itemize}
        \item Keyword: KEY
        \item Plaintext: ATTACK AT DAWN
        \item Ciphertext: KXRKGI EX HEUR
    \end{itemize}
\end{itemize}
\end{frame}

\begin{frame}{Vigènere Cipher Example}
\begin{itemize}
    \item Keyword: KEY
    \item Plaintext: ATTACK AT DAWN
    \item Ciphertext: KVVCCS DC HCPZ
\end{itemize}
\end{frame}

\begin{frame}{Build Your Own Vigenère Cipher}
\begin{itemize}
    \item Choose a keyword
   \item Encrypt the message
   \item Example: "HELLO WORLD" with "KEY"
   \item Ciphertext: \_ \_ \_ \_ \_ \ \_ \_ \_ \_ \_ \_
\end{itemize}
\end{frame}

\begin{frame}{Playfair Cipher}
\begin{itemize}
    \item Digraph substitution cipher
    \item Uses a 5x5 matrix from keyword
    \item Encrypts pairs of letters
    \item Example:
    \begin{itemize}
        \item Keyword: MONARCHY
        \item Plaintext: HELLO
        \item Ciphertext: HF LX PR
    \end{itemize}
\end{itemize}
\end{frame}

\begin{frame}[fragile]{Playfair Cipher Matrix}
\begin{tabular}{|c|c|c|c|c|}
  M & O & N & A & R   \\
  C & H & Y & B & D   \\
  E & F & G & I/J & K \\
  L & P & Q & R & S   \\
  T & U & V & W & X   \\
\end{tabular}
\end{frame}

\begin{frame}{Enigma}
    \begin{itemize}
        \item In World War II, both Germany and Japan relied on electromechanical machines to encrypt messages
        \item Germany: The Enigma machine
        \item Japan: similar Type B (US called it Purple)
        \item By standard of the times, cipher was sophisticated 
    \end{itemize}
    \includegraphics[scale=0.25]{wikipedia_enigmamachine.jpg}
    \end{frame}

\begin{frame}{Enigma Details}
    \begin{itemize}
        \item Each character typed rotates the rotor
        \item Every full rotation, the next rotor advances by 1
        \item User selects 3 rotors from a kit of 5
        \item There is also a plugboard swapping pairs of letters
        \item Not enough to capture a machine, the settings of the rotors and plugboard are required
    \end{itemize}
\end{frame}

\begin{frame}{Breaking Enigma}
\begin{itemize} 
    \item UK attacked Enigma messages by using
    \begin{itemize}
        \item Electromechanical computer simulating 36 enigmas in parallel (Bombe)
        \item Shown in movie \url{https://en.wikipedia.org/wiki/The_Imitation_Game}
        \item Later electronic machine computing faster with vaccuum tubes
    \end{itemize}
    \item Plaintext attacks weather messages which were rigidly formatted
    \item Messages enciphered both in enigma and another breakable method
\end{itemize}
\includegraphics[scale=0.25]{bombe.jpg}
\includegraphics[scale=0.25]{Colossus.jpg}
\end{frame}

\end{withoutheadline}

\end{document} 
